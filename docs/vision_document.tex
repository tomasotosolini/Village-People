\subsection{High level project goals} 
The system will simulate life in a town. It will simulate people living in and
services they interact with in everyday life. 
To achieve this goal the system will simulate time passing by and a Simulated
Environment (SIM) to provide the entities with access to resources, and to
schedule events in both predictable and random way.
The entities involved in the simulation (like simulated people and simulated
services) will interact with the Simulated Environment both actively and
passively. 
In the active interaction the entity will affect the simulated environment in
some way (for example consuming a resource). 
In the passive interaction the entity will be affected by a simulated
environment decision (for example an entity destruction action).
Simulated Entities will interact between themselves independently from the
Simulated Environment. The interaction will be initiated by the Communication
Coordinator which decides who will be part of it and its role. The
interaction will then continue independently from the Communication Coordinator
and the outcome will only depend on its participants. The interaction's
termination can be due to logic defined into the interaction itself or can be
due to external actions of other System Entities.

\subsection{Core Resources} 
\begin{enumerate}
\item Simulation Resource (SCR)
\item Time Resource (TCR)
\item Communication Coordinator (CCR)
\end{enumerate}

\subsection{Core Simulation Entities} 
\begin{enumerate}
\item Entity Interaction (EI)
\end{enumerate}

\subsection{Core Simulated Entities} 
\begin{enumerate}
\item People (PCE)
\item Houses (HCE)
\end{enumerate}

\subsection{Core Simulated Services} 
\begin{enumerate}
\item People Id Registry (PIRS)
\item Hospital (HS)
\item Bank (BS)
\end{enumerate}

\subsection{Consumables} 
\begin{enumerate}
\item Fuel (FUC)
\item Food (FOC)
\end{enumerate}

\subsection{Exchangeables} 
\begin{enumerate}
\item Money (MONEY)
\end{enumerate}


\subsection{System core use cases}
\begin{itemize}
  \item \textbf{Time passing Cycle}  
	
	- Actors: Time, Time Affected Entity\footnote{formerly TAE}

	- Time will tick and every affected entity is informed about the current
simulated universal time.

  \item \textbf{Simulated Environment Cycle}

	- Actors: Simulated Environment, Entities

	- As a time affected entity, Simulated Environment will update its
status and the status of its Simulated Resources (formerly SR) and regulate
access to them. Moreover it will also trigger both scheduled and random events.

  \item \textbf{Resource Usage}

	- Actors: Resource, Consumer

	- The consumer asks the Simulated Environment for a specific amount of a
specific resource, the amount is either granted or a 'no more available' message
is sent back to the consumer.

  \item \textbf{People Life} 

	- Actors: Simulation, Person, Time

	- Simulation creates Person then Person lives and after some time the
Simulation destroys the Person

  \item \textbf{People Id Registry (PIR) Enquiry}

	- Actors: People Id Registry, other System's Entity

	- A System's Entity enquiries the PIR to obtain information about some
other System's Entity. The PIR will then supply the requested information  if
available

  \item \textbf{People Id Registry (PIR) Birth Registration}

	- Actors: People Id Registry, Simulation, Person

	- Once the Simulation creates a new Person, it also asks to PIR to
register the birth event

  \item \textbf{People Id Registry (PIR) Death Registration}

	- Actors: People Id Registry, Simulation, Person

	- Once the Simulation destroy a new Person, it also asks to PIR to
register the death event

  \item \textbf{Hospital (H) Request}

	- Actors: Person, Hospital

	- A Person goes to the Hospital, and after some time exits in three
possible states: healed, not healed, death

  \item \textbf{Bank (B) Fund Request}

	- Actors: Person, Bank

	- A Person goes to the Bank and obtains some fund

  \item \textbf{Bank (B) Fund Depot}

	- Actors: Person, Bank

	- A Person goes to the Bank and depots some fund
\end{itemize}


\subsection{High level architecture}
Service oriented architecture. The components must be organized into independent
modules. It is foreseen that the client access is multi-protocol. Also the
services will be able to exchange message in a client-server mode.

It appears reasonable to choose a component based development model because
of the intended modular structure of the software parts involved in to system.
Moreover the component modelling encourages the separation of the various parts
of the structure and leads to a service oriented design, potentially remote
services.

\subsection{Project risks overview}
Divert the development focus from developing a simulation to develop the
simulated entity.
